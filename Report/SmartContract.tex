\documentclass[runningheads,a4paper]{llncs}

\usepackage{amssymb}
\usepackage{amsmath}
\setcounter{tocdepth}{3}
\usepackage{graphicx}

\usepackage{url}
\urldef{\mailsa}\path|a.bruno25@studenti.unipi.it|    
\newcommand{\keywords}[1]{\par\addvspace\baselineskip
\noindent\keywordname\enspace\ignorespaces#1}

\begin{document}

\mainmatter  % start of an individual contribution

% first the title is needed
\title{Smart Auction}
\subtitle{Dutch and Vickrey}

\author{Andrea Bruno}

\institute{
\emph {Department of Computer Science} \\
University of Pisa\\
\mailsa}
\maketitle


\section{Introduction}
We were asked to implement two types of auction systems using the Ethereum blockchain. The first auction system is called Vickery whereas the second one (chosen by me) is called Dutch. Both systems share a so called ``grace period'' that is a time window of 5 minutes, in which the auction exists but is not active. In this implementation of the system, the common features of the two auctions are placed into an abstract contract called \url{Auction.sol}, whereas the specifics operations are left to the ``sub-contracts'' \url{DutchAuction.sol} and \url{VickreyAuction.sol}.


\section{Auction.sol}
To define a skeleton of shared functionalities, I chose to exploit the \emph{Template-Method pattern}. This pattern requires a base class defining the shared code and some subclasses who will implement specific logics.


As you can see from Figure \ref{fig:auction}, in the Auction contract there is a public \url{struct} called \emph{Description} that simulate the price-tag of the item that will be sold. This way, the basic info will be placed on the blockchain, hence people can watch the item without consuming gas, like a real shop.

As far as the ``grace period'' concerned, this functionality has been described within the \url{activateAuction} method, but since it is an abstract method, the implementation is delegated to the subclasses who will implement the Auction class. There is also a \url{finalize} method that allow the contract to be closed and confirmed adequately. Another fundamental part is the \url{onlySeller} modifier. This latter, when declared alongside a function, guarantee us that only authorized people (in this case the seller) can run that function. Moreover to communicate the begin and the end of the auction, two events have been declared.


\begin{figure}[h]
\includegraphics[width=\linewidth,]{images/auction.png}
\centering
\caption{Source code of \emph{Auction.sol}}
\label{fig:auction}
\end{figure}


\newpage
\section{DutchAuction.sol}
As specified in the requirements, the Dutch auction consists of an initial phase in which the price of the good is set to a very high price and afterwards, is gradually lowered by following a precise strategy. The auction finishes when someone is willing to pay the price of the good.

In this implementation of the Dutch auction, the contract extends the abstract contract \url{Auction} in order to  follow the \emph{Template-Method pattern} correctly and additionally, it uses a contract of class \url{Strategy} to implement the logic of devaluation.

\begin{figure}
\includegraphics[width=140pt]{images/dutchUML.png}
\centering
\caption{UML view of \emph{DutchAuction}}
\label{fig:dutchUML}
\end{figure}



\newpage
\subsection{Strategy.sol \protect\footnote{Since every Dutch auction requires a strategy, we first describe the Strategy contract and afterwards the DutchAuction implementation.}}

We were asked to develop multiple method to compute the decrease of the price. In this work, this requirement has been developed exploiting an abstract contract (Figure \ref{fig:strategycode}) containing only one abstract method called \url{getPrice}.

\begin{figure}[h]
\includegraphics[width=\linewidth]{images/strategycode.png}
\centering
\caption{Source code of \emph{Strategy} abstact class}
\label{fig:strategycode}
\end{figure}

Afterwards, we create three sub-contracts implementing the \url{Strategy} contract, in order to let the user choice his own favorite strategy. In particular, \url{NormalStrategy} decreases in a linear way, then \url{SlowStrategy} decreases twice slower than \url{NormalStrategy} and finally, \url{FastStrategy} decreases twice faster than \url{NormalStrategy}.

\begin{figure}[h]
\includegraphics[width=220pt]{images/strategyUML.png}
\centering
\caption{UML view of the multiple \emph{Strategy}}
\label{fig:strategyUML}
\end{figure}

\newpage
\subsection{Status} \label{status}
Although in the Dutch auction there are only three explicit phases, for security reasons we added a new phase called \emph{Verification}. Since there could be the case in which two people make simultaneously a bid, as Figure \ref{fig:dutchStatus} shows, before closing the contract, we verify that at least 12 blocks \cite{ethBlog} have been confirmed. This restriction is essential to ensure that all miners own the same status of the contract and thus the same winner.

\begin{figure}[h]
\includegraphics[width=220pt]{images/dutchStatus.png}
\centering
\caption{Status of the \emph{Dutch} auction}
\label{fig:dutchStatus}
\end{figure}


\subsection{Constructor}
To create a new Dutch auction you need to insert the name of the item that you are willing to sold, both a reserve price and an initial price and lastly the address of a Strategy contract.

\begin{figure}[h]
\includegraphics[width=0.6\linewidth]{images/DutchConstructor.png}
\centering
\caption{Source code of the \emph{DutchAuction} constructor}
\label{fig:dutchConstructor}
\end{figure}

Once deployed the contract correctly, its own status is set to \emph{Grace Period} . Note that in this implementation, the role of auctioneer does not exist and all critical functionalities are left to the seller of the item.

\newpage
\subsection{Functions}

\begin{figure}[h]
\includegraphics[width=\linewidth]{images/DutchCode.png}
\centering
\caption{Source code of the \emph{DutchAuction} functions}
\label{fig:DutchCode}
\end{figure}

To activate the auction, the \emph{Grace Period} of 5 minutes should finish. According to \url{etherscan}\cite{etherscan} every 15 seconds a new block is confirmed, hence:


\[5 \, min = 300 \, sec \implies \frac{300 \, sec}{15 \, sec/block}  = 20 \, blocks\]

By waiting 20 blocks we can (more or less) ensure the 5 minutes asked by the requirements.

Another interesting aspect is the \url{getActualPrice} function. Since the price drops over time, it must be updated, thus a public \url{uint} value on the blockchain is not sufficient. For this reason, people interested in knowing the price, should pay some gas to update the price. 

Now let's analyze the \url{bid} function. At the beginning of the function, we require that the \url{msg.value} (in other words, the money sent) are greater or equal the \url{actualPrice}. Right after this control, it is immediately written on the blockchain both the address of the user that made the bid, and the value of ether that he sent to the contract. Afterwards, due to security reasons related to simultaneous bids (as previously discussed in \ref{status}), the contract passes in a \emph{Validating} status. From now on, the seller can finalize the contract and receive the ether he's owed. 


\subsection{Gas Evaluation}

\begin{table}
\setlength{\tabcolsep}{10pt}
\setlength{\abovecaptionskip}{10pt}
\centering
\begin{tabular}{| c | c | c | c |} 
 \hline
 \textbf{Function} & \textbf{Transaction Cost} & \textbf{Execution Cost} & \textbf{Caller} \\
 \hline
 \emph{constructor} & 1067764 & 787072 & Seller \\ 
 \hline
 \url{activateAuction()} & 63359 & 42087 & Seller \\ 
 \hline
 \url{getActualPrice()} & 29677 & 8405 & All \\ 
 \hline
 \url{bid()} & 95839 & 74567 & All \\ 
 \hline
 \url{finalize()} & 37983 & 16711 & Seller \\
 \hline
 \hline
 \hline
 \textbf{\emph{Total}} & 1294622 & 928842 & \\ \hline
\end{tabular}
\caption{Gas consumption of \emph{DutchAuction} contract}
\label{table:1}
\end{table}




\newpage
\section{VickreyAuction.sol}




\section{The References Section}\label{references}
The following section shows a sample reference list with entries for
journal articles \cite{jour}, an LNCS chapter \cite{lncschap}, a book
\cite{book}, proceedings without editors \cite{proceeding1} and
\cite{proceeding2}, as well as a URL \cite{ethBlog}.
Please note that proceedings published in LNCS are not cited with their
full titles, but with their acronyms!

\begin{thebibliography}{4}

\bibitem{jour} Smith, T.F., Waterman, M.S.: Identification of Common Molecular
Subsequences. J. Mol. Biol. 147, 195--197 (1981)

\bibitem{lncschap} May, P., Ehrlich, H.C., Steinke, T.: ZIB Structure Prediction Pipeline:
Composing a Complex Biological Workflow through Web Services. In: Nagel,
W.E., Walter, W.V., Lehner, W. (eds.) Euro-Par 2006. LNCS, vol. 4128,
pp. 1148--1158. Springer, Heidelberg (2006)

\bibitem{book} Foster, I., Kesselman, C.: The Grid: Blueprint for a New Computing
Infrastructure. Morgan Kaufmann, San Francisco (1999)

\bibitem{proceeding1} Czajkowski, K., Fitzgerald, S., Foster, I., Kesselman, C.: Grid
Information Services for Distributed Resource Sharing. In: 10th IEEE
International Symposium on High Performance Distributed Computing, pp.
181--184. IEEE Press, New York (2001)

\bibitem{proceeding2} Foster, I., Kesselman, C., Nick, J., Tuecke, S.: The Physiology of the
Grid: an Open Grid Services Architecture for Distributed Systems
Integration. Technical report, Global Grid Forum (2002)

\bibitem{ethBlog} Ethereum Blog, 
\url{https://blog.ethereum.org/2015/09/14/on-slow-and-fast-block-times}

\bibitem{etherscan} Block Explorer and Analytics Platform for Ethereum, 
\url{https://etherscan.io/chart/blocktime}

\end{thebibliography}


\section*{Appendix: Springer-Author Discount}

LNCS authors are entitled to a 33.3\% discount off all Springer
publications. Before placing an order, the author should send an email, 
giving full details of his or her Springer publication,
to \url{orders-HD-individuals@springer.com} to obtain a so-called token. This token is a
number, which must be entered when placing an order via the Internet, in
order to obtain the discount.


\end{document}